\documentclass[12pt]{article}

\usepackage{paper}

%commands
\renewcommand{\thefootnote}{\fnsymbol{footnote}}
\renewcommand{\theequation}{\arabic{equation}}

\date{}

\begin{document}
{\Large\noindent
[Brief Review\footnote{
The reviewer: Daichi Takeda (takedai.gauge@gmail.com)
}]\\[2mm]
%title
\textbf{Light-cone cuts and hole-ography:\\explicit reconstruction of bulk metrics
\cite{Takeda:2021dsl}
}
}

\noindent
\hfill
\textbf{Daichi Takeda}%author

\renewcommand{\thefootnote}{\arabic{footnote})}
\setcounter{footnote}{0}
\vspace{12pt}
%%body
\subsection*{Motivation and Summary}
In this paper, I showed that there is a class of AdS$_3$/CFT$_2$ theories where bulk metrics are completely
reconstructed from boundary entanglement entropy.
We use the light-cone cuts method \cite{Engelhardt:2016wgb} and 
the hole-ography on points and distances \cite{Czech:2014ppa}.

The light cone-cuts method reconstructs the bulk causal structure from divergent correlators of the boundary.
In other words, the method reconstructs the conformal metric (the metric up to a conformal rescaling).
The method can be applied to the most generic case of all the methods known so far,
thus finding a generic way to recover the conformal factor is an important subject,
which is still under investigation (several attempts are seen in \cite{Engelhardt:2016crc}).
I have written a review of the light-cone cuts method
\hyperlink{https://albertmcc.github.io/web/reviews/light_cone1.pdf}{\color{blue}{here}}.

I considered that with the help of the hole-ography, the conformal factors can be determined.
The hole-ography gives a holographic definition of points and distances on a time slice of AdS$_3$,
in terms of the entanglement entropy of the boundary CFT$_2$.
As was conjectured by the authors in \cite{Czech:2014ppa},
their formulation seems valid at most for static AdS$_3$/CFT$_2$ theories in a rotationally symmetric state\footnote{
The boundary theory is on a cylinder and its coordinate is $(t,\theta)$ with $\d s^2 = -\d t^2 + L^2\d\theta^2$.
Here $L$ is the AdS radius.
the rotational symmetry means $\theta$-translation here.
}.
I hope that 
\hyperlink{https://albertmcc.github.io/web/reviews/Nuts_and_Bolts_for_Creating_space.pdf}{\color{blue}{my review}}
would be helpful to overview the method.

To examine if the hole-ography helps us recover the undetermined conformal factors,
let us consider static AdS$_3$/CFT$_2$ theories in a rotationally symmetric state,
where the hole-ography works well.
Though the guess naively appears to work, there is a problem.
The information about bulk points is encoded in different boundary quantities in different reconstruction methods;
this is the case here.
In the light-cone cuts method, the quantities are light-cone cuts, and in the hole-ography,
the quantities are point functions.\footnote{
You can check the definitions of light-cone cuts and point functions in each review introduced above.
}
Thus, we have to find the proper one-to-one map between light-cone cuts and point functions.

Unfortunately, I could not find the generic relation between them.
However, I found the specific case where there is a direct relation between light-cone cuts and hole-ography.
The case is when entanglement wedge and causal wedge coincide for boundary intervals.\footnote{
When the bulk is a black hole geometry,
an entanglement wedge wraps the black hole, if the corresponding boundary interval is large.
We can ignore such large intervals, since they have nothing to do with the reconstruction procedure.
}
If we assume it, the bulk metric is completely reconstructed by my method.


\subsection*{Future direction}
\begin{itemize}
	\item To find the general map between light-cone cuts and point functions.
	\item To extend the prescription to higher dimensions.
	\item To examine the inside of the black hole.
\end{itemize}



%%
\bibliographystyle{jhep} 
\bibliography{reference.bib}




























\end{document}
