\documentclass[12pt]{article}

\usepackage{paper}

%commands
\renewcommand{\thefootnote}{\fnsymbol{footnote}}
\renewcommand{\theequation}{\arabic{equation}}

%title,author
\title{\textbf{Review}\\
Interior Product, Lie derivative and Wilson Line in the \textit{KBc} Subsector of Open String Field Theory
}
\author{Hiroyuki Hata and Daichi Takeda}
\date{}
\begin{document}
{\Large\noindent
[Brief Review\footnote{
The reviewer: Daichi Takeda (takedai.gauge@gmail.com)
}]\\[2mm]
\textbf{Interior Product, Lie derivative and Wilson Line in the \textit{KBc} Subsector of Open String Field Theory
}
}

\noindent
\hfill\textbf{Hiroyuki Hata and Daichi Takeda}

\renewcommand{\thefootnote}{\arabic{footnote})}
\setcounter{footnote}{0}
\vspace{12pt}
%%body
\textit{KBc} sector is a subspace of bosonic open string field theory (SFT) expanded by
\begin{align}
	K = \int_{-i\infty}^{i\infty}\frac{\d z}{2\pi i}T(z),\quad
	B =  \int_{-i\infty}^{i\infty}\frac{\d z}{2\pi i}b(z),\quad 
	c = c(0),
\end{align}
where $T$ is the energy-momentum tensor, $b$ is the anti-ghost field, and $c$ is the ghost field.
Here operators are defined in a coordinate called sliver frame $z$, 
which is defined through the map $z = (2/\pi)\arctan w$ with the coordinate $w$ used in radial quantization.
The following relations called \textit{KBc} algebra are satisfied by $K,B,c$ and BRST operator $Q_\mathrm{B}$:
\begin{align}
	&\cmt{K}{B} = 0,\quad
	\acmt{B}{c} = 1,\quad
	B^2 = 0,\quad
	c^2 = 0,\\
	&Q_\mathrm{B}K = 0,\quad
	Q_\mathrm{B}B = K,\quad
	Q_\mathrm{B}c = cKc.
\end{align}
Since any CFT has \textit{KBc} sector, any solution in \textit{KBc} sector is universal.
Other Analytic solutions are studied by combining \textit{KBc} sector with some matter operators,
where solutions are not universal in general.
The review paper \cite{Okawa:2012ica} is useful to learn \textit{KBc} sector and classical solutions including tachyon vacuum.

We established the notion of the manifold in \textit{KBc} sector in the paper.
A top-down approach is taken in this review, while a heuristic approach is in the paper.
Let $\xi = (\xi^1,\xi^2)$ a two-component function, $\xi:\mathbb R\to \mathbb C^2$.
Then it is easily confirmed that the new triad $(K(\xi),B(\xi),c(\xi))$ defined as follows again forms \textit{KBc} algebra:
\begin{align}
	K(\xi) := e^{\xi^1(K)}K,\quad B(\xi) := e^{\xi^1(K)}B,\quad c(\xi) = e^{-i\xi^2(K)}ce^{-\xi^1(K)}Bce^{i\xi^2(K)}.
\end{align}
We define \textit{KBc} manifold $\mathcal K$ as follows:
\begin{itemize}
	\item The triad $(K(\xi),B(\xi),c(\xi))$ is a point of $\mathcal K$.
	\item The coordinate is the function $\xi$.
\end{itemize}

Next, let us find a suitable definition of the interior product $\mathcal I_X$, 
where $X = (X^1,X^2)$ is a two-component function.
It is known that the action of bosonic open SFT (called Witten's action) has similar structure to the Chern-Simons (CS) action.
Especially, the ghost number in bosonic open SFT corresponds to the form number in CS theory.
Therefore, we assume that $\mathcal I_X$ lower the ghost number by one.
In addition, \textit{KBc} algebra should not be broken by the operation of $\mathcal I_X$ at each point of $\mathcal K$.\footnote{
For example, $\mathcal I_X(\{B,c\}) = \mathcal I_X (1)$ should hold for the relation $\{B,c\} = 1$ not to be broken by introducing $\mathcal I_X$ to \textit{KBc} sector.
}
Under the two conditions,\footnote{
We imposed more conditions in the original paper, for example nilpotency, 
but now it was turned out that the two condition are sufficient to define our interior product.}
the general form of the operation of $\mathcal I_X$ at $\xi$ is given by\footnote{
Here $X^{1,2}$ can depend on the original $K$ not through $K(\xi)$, and the factor $1/K(\xi)$ accompanying $X^2$ is just for
 convenience to simplify the expression of the Lie derivative introduced in the next paragraph.
}
\begin{align}
	\mathcal I_X K(\xi)= iB(\xi)X^1(K),\quad
	\mathcal I_X B(\xi) = 0,\quad
	\mathcal I_X c(\xi) = \l\{\frac{X^2(K)}{K(\xi)}B(\xi),c(\xi)\r\}.\label{eq:interior}
\end{align}
We further impose on the interior product Leibniz-rule
\begin{align}
	\mathcal I_X (PQ) = (I_X P)Q + (-1)^{|P|}P(\mathcal I_X Q),\label{eq:Leibniz}
\end{align}
for the operation of $\mathcal I_X$ to be defined for any quantities in \textit{KBc} sector.
Here $|P|$ is $0$ ($1$) when $P$ is Grassmann-even (odd).

By analogy with the ordinary manifold ($L_V = \{\mathrm{d},i_X\}$), we define the Lie derivative as
\begin{align}
	\pounds_X = -i\{Q_\mathrm{B},\mathcal I_X\}.
\end{align}
We have used a known fact that BRST operator corresponds to the exterior derivative in the similarities described above.
The concrete forms of the operation of $\pounds_X$ at $\xi$ are given by
\begin{align}
	&\pounds_X K(\xi) = K(\xi)X^1(K),\quad
	\pounds_X B(\xi) = B(\xi)X^1(K),\nonumber\\
	&\pounds_X c(\xi) = -c(\xi)X^1(K)B(\xi)c(\xi) - i[X^2(K),c(\xi)].
\end{align}
From the definition and  \eqref{eq:Leibniz}, $\pounds_X$ follows Liebniz-rule
\begin{align}
		\pounds_X (PQ) = (\pounds_X P)Q + P(\pounds_X Q).
\end{align}

The ordinary properties and formulas satisfied by the ordinary interior product, exterior product and Lie derivative 
still hold here with the exterior derivative replaced with BRST operator.
For example, we have $\{I_X,I_Y\} = 0$, $[Q_B,\pounds_X] = 0$ and $[\pounds_X,\pounds_Y] = \pounds_{[X,Y]}$, 
where $[X,Y]$ is a Lie bracket properly defined in the original paper.
As other benefits, the followings are introduced in the paper.
\begin{itemize}
	\item The \textit{KBc} triad at $\xi + \delta \xi$ can be obtained by acting $\pounds_{\delta \xi}$ on the triad at $\xi$ 
		for any infinitesimal $\delta\xi$. Then any two points on a continuous curve on $\mathcal K$ are related by 
		the continuous operation of Lie derivative.
	\item If $\Psi$ is a solution in \textit{KBc} sector, 
		then $\Psi(\xi)=\Psi|_{(K,B,c)\to (K(\xi),B(\xi),c(\xi))}$ is also a solution.
	\item Wilson line is also defined by analogy with CS theory. Although some similar properties to the ordinary ones still hold,
		 the construction is incomplete.
\end{itemize}
Possible future works are as follows.
\begin{itemize}
	\item The physical interpretation of \textit{KBc} manifold.
	\item Improving the Wilson line.
	\item The extension to the remaining sector of bosonic open SFT.
	\item The case of super SFT.
\end{itemize}

%We addressed how to emerge Chan-Paton factors around the multi-$\mathrm{D}p$ brane solution found by Murata and Schnabl \cite{Murata:2011ex}.
%Comparing with the action with Chan-Paton factors, the fluctuation $\Delta\Psi$ around the solution of $N$ branes is expected to expanded by the modified form of vertex operators $V_a^\ddag\mathcal O_F(k) V_b$  with some operators $V_a\,(a = 1,\cdots,\,N)$ carrying no ghost.
%Then we find that the question becomes whether there exist $N$ operators $V_a$ satisfying
%\begin{align}
%	V_a(\overleftarrow Q_\mathrm{B}+\Psi_0) = 0\quad
%	\mathrm{and}\quad 
%	V_aV_b^\ddag = \delta_{ab},\label{eq:Va_coditions}
%\end{align}
%where $\Psi_0$ is the solution of $N$ branes and $A\overleftarrow Q_\mathrm{B} = -(-1)^{|A|}Q_\mathrm{B}A$.
%The similar equation for the former can be seen in Wilson lines of gauge theories, and the formal resemblance between SFT and Chern-Simons theory has been known,
%so we are driven to find the correspondence of Wilson line in SFT.
%
%In the process to find this, we established the \textit{KBc} manifold, which includes infinite sets of distinct $(K,B,c)$, of course satisfying \textit{KBc} algebra, as its points.
%On the manifold, we succeeded in finding the interior product, Lie derivative and Lie bracket.
%The Lie derivative gives infinitesimal deformations in the space of solutions.
%By using the interior product, we finally obtained a candidate for the SFT Wilson line.
%However, the task to solve \eqref{eq:Va_coditions} is still left, and the information about \textit{KBc} manifold may be not enough.
%For example, formulas for the interior product and the Lie derivative in the SFT side are slightly different from in the Chern-Simons side.

%%%%%
\bibliographystyle{jhep} 
\bibliography{reference.bib}
\end{document}
