\documentclass[12pt]{article}

\usepackage{paper}

%commands
\renewcommand{\thefootnote}{\fnsymbol{footnote}}
\renewcommand{\theequation}{\arabic{equation}}
\date{}

\begin{document}
{\Large\noindent
[Brief Review\footnote{
The reviewer: Daichi Takeda (takedai.gauge@gmail.com)
}]\\[2mm]
%title
\textbf{Towards a Reconstruction of General Bulk Metrics\cite{Engelhardt:2016wgb}
}
}

\noindent
\hfill
\textbf{N. Engelhardt and G.T. Horowitz}%author

\vspace{12pt}
%%body
A new approach for the bulk reconstruction, the method using the light-cone cuts, is introduced.
The past (future) light-cone cut of the bulk point $p$ is defined as
\begin{align}
	C^-(p) = \partial J^-(p)\cap \partial M~(C^+(p) = \partial J^+(p)\cap \partial M),
\end{align}
where $M$ is the spacetime assumed to be asymptotically $\mathrm{AdS}_n$ with some natural conditions 
for mathematical reasons.
From now on, we focus on past cuts, but the same discussion can be applied to future cuts.

First, using the ``bulk-point singularity" \cite{Maldacena:2015iua}, we can get the past cuts of the points 
in $J^+(\partial M)\cap J^-(\partial M)$.
A certain class of divergence of boundary to boundary massless correlators,
is caused by the vertex being in the bulk at which total null momenta is conserved.
Such a divergence is called bulk-point singularity.
Let us consider a boundary correlator consisting of $n$ points $x_1,\cdots,x_n$ which are spacelike separated each other, 
and two points $z_1, z_2$ in the past of $x_i$:
\begin{align}
	\langle\mathcal O(z_1)\mathcal O(z_2)\mathcal O(x_1)\cdots\mathcal O(x_n)\rangle.
\end{align}
If we move $z_1$ and $z_2$ as the correlator diverges while keeping the momentum conservation,
we see that $z_1$ and $z_2$ draw the past cut of bulk point null-separated from $x_i$.
Since we do not know about the bulk geometry, we cannot know whose past cut it is.
In principle, however, the past cut can be labeled by some $n$ parameters related to $x_i$.
Let us set the parameters $\lambda = (\lambda^1,\cdots,\lambda^n)$ and denote the cut by $C^-_\lambda$.
By repeating this step, all past cuts related one-to-one to points in $J^+(\partial M)\cap J^-(\partial M)$ can be collected, 
then we define the vector space of $\lambda$'s, $\mathcal M^- := \{\lambda\}$.

Let $\lambda_1,\lambda_2,\cdots$ be vectors on $\mathcal M^-$, 
meaning that $C^-_{\lambda_1},C^-_{\lambda_2}\cdots$ are the past cuts.
In the appendix, it is shown that, if $C^-_{\lambda_1} = C^-(p)$ and $C_{\lambda_2}^- = C^-(q)$ are tangent each other 
precisely at one point, then $p$ and $q$ are null related.
Therefore, taking a boundary point $x \in C^-_{\lambda_1}$ and collecting the past cuts tangent to $C^-_{\lambda_1}$ at $x$,
 the unique null geodesic from $p$ to $x$ via $q$ is obtained.
Then we can identify this null geodesic to the one connecting $\lambda_1$ and other $\lambda$'s corresponding to 
other cuts tangent to $C^-_{\lambda_1}$ at $x$, because of the existence of the one-to-one map from 
$J^+(\partial M)\cap J^-(\partial M)$ to $\mathcal M^-$.
Repeating this process for points on $C^-_{\lambda_1}$, we get null geodesics through $\lambda_1$, 
and they form light-cone of $\lambda_1$ in $\mathcal M^-$.

Once the causal structure is introduced in $\mathcal M^-$, the conformal metric (the metric up to conformal factors) 
can be determined.
The null generators at $\lambda\in \mathcal M^-$ are obtained by looking for other past cuts tangent to $C^-_{\lambda}$.
Let $\sigma = (\sigma^1,\cdots,\sigma^{n-2})$ be the parameter describing the points on 
$C^-_{\lambda}$ as $C^-_{\lambda}(\sigma) \in \partial M$.
The condition which $C^-_{\lambda_2}$ is tangent to $C^-_{\lambda_1}$ is as follows:
\begin{align}
	\exists\sigma_1,~\exists\sigma_2,\quad C^-_{\lambda_1}(\sigma_1) = C^-_{\lambda_2}(\sigma_2)\quad\mathrm{and}\quad \l.\nabla_{\sigma} C^-_{\lambda_1}(\sigma)\r|_{\sigma = \sigma_1} = \l.\nabla_{\sigma} C^-_{\lambda_2}(\sigma) \r|_{\sigma = \sigma_2}.
\end{align}
A null generator $N$ at $\lambda$ satisfies the above condition for $\lambda_1 = \lambda$ and 
$\lambda_2 = \lambda + \varepsilon N$, where $\varepsilon$ is any infinitesimal real variable.
By solving the conditions, we can find out $n(n+1)/2$ generators at $\lambda$.
Imposing the condition that the $n(n+1)/2$ null generators are null at $\lambda$ determines the metric at $\lambda$ up to
 conformal factors.
Since $\mathcal M^-$ can be regarded as the copy of $J^+(\partial M)\cap J^-(\partial M)$, the reconstruction of
 the bulk metric has been completed now.


%%
\bibliographystyle{jhep} 
\bibliography{reference.bib}
\end{document}
