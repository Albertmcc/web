\documentclass[12pt]{article}

\usepackage{paper}

%commands
\renewcommand{\thefootnote}{\fnsymbol{footnote}}
\renewcommand{\theequation}{\arabic{equation}}

\date{}

\begin{document}
{\Large\noindent
[Brief Review\footnote{
The reviewer: Daichi Takeda (takedai.gauge@gmail.com)
}]\\[2mm]
%title
\textbf{Holographic Reconstruction of Bubbles\cite{Burda:2018rpb}
}
}

\noindent
\hfill
\textbf{Philipp Burda, Ruth Gregory, Akash Jain}%author

\renewcommand{\thefootnote}{\arabic{footnote})}
\setcounter{footnote}{0}
\vspace{12pt}
%%body
This paper examined whether the two bulk reconstruction methods, the hole-ography method  \cite{Balasubramanian:2013lsa,Czech:2014ppa}, and the light-cone cuts method \cite{Engelhardt:2016wgb} , are valid for BTZ black hole and BTZ bubble the metrics of which are introduced below.
In the each case, the boundary observable used in the reconstruction is originally calculated in the boundary, but it is calculated by using bulk geodesics; we assume the bulk metric first, and once the boundary observable is obtained, we forget the bulk metric in the reconstruction process.
	Therefore, their work is to confirm whether these methods are valid, assuming that we know the boundary observables.

BTZ black hole without charge and angular momentum is a $(2+1)$ dimensional black hole spacetime which asymptotes to $\mathrm{AdS}_3$:
\begin{align}
	\d s^2 = \frac{1}{z^2}\l[-(1-Mz^2)\d t^2 + \d x^2 + \frac{\ell^2}{1-Mz^2}\d z^2\r].
\end{align}
BTZ bubble metric is given by
\begin{align}
	\d s^2 = 
	\l\{
		\begin{array}{cc}
			\displaystyle \frac{1}{z^2}\l[-(1-M_+z^2)\d t^2 + \d x^2 + \frac{\ell_+^2}{1-M_+z^2}\d z^2\r] & \mathrm{for}~z\le Z(\tau),\\[5mm]
			\displaystyle\frac{1}{z^2}\l[-(1-M_-z^2)\d t^2 + \d x^2 + \frac{\ell_-^2}{1-M_-z^2}\d z^2\r] & \mathrm{for}~z\ge Z(\tau),
		\end{array}
	\r.
\end{align}
where $z = Z(\tau)$ is  a timelike hypersurface with timelike coordinate $\tau$, and called bubble.
From some suitable conditions, they require
\begin{align}
	M_+ > M_-,\qquad \frac{\ell_+}{M_+} \geq \frac{\ell_-}{M_-}.
\end{align}

In the hole-orgraphy method spacelike geodesics are used to calculate Ryu-Takayanagi surfaces (1dim lines), and using Ryu-Takayanagi formula, the entanglement entropy $S(\alpha)$ is obtained as a function of the angle interval $\alpha$ on the boundary.
The hole-ography method can reconstruct the points and their distances of the pure $\mathrm{AdS}_3$ bulk, but for the sake of applying it to other asymptotic $\mathrm{AdS}_3$ metric, the following is conjectured in \cite{Czech:2014ppa}.
Each point on the time slice corresponds to a certain boundary function of the angular coordinate $\alpha(\theta)$, the support of which is the intersection of the time slice and the boundary.
Function $\alpha(\theta)$ is the substitution of a bulk point, if and only if $\alpha(\theta)$ extremizes the action\footnote{The strong subadditivity of the entropy ensures $-\d^2S/d\alpha^2 > 0$.}
\begin{align}
	I[\alpha] = \int_0^{2\pi}\d\theta \sqrt{-\l. \frac{\d^2S}{\d\alpha^2}\r |_{\alpha = \alpha(\theta)}(1-\alpha'(\theta)^2)}~.
\end{align}
The Eular-Lagrange equation reads
\begin{align}
	2\alpha''(\theta)\l.\frac{\d^2S}{\d\alpha^2}\r|_{\alpha = \alpha(\theta)} + (1-\alpha'(\theta)^2)\l.\frac{\d^3S}{\d\alpha^3}\r|_{\alpha = \alpha(\theta)} = 0.
\end{align}
However, it turns out that this conjecture does not hold for BTZ bubble, while it hold for BTZ.
They concluded that the hole-ography method isn't a generic one to reconstruct bulks.\footnote{I think there still remains a possibility that Ryu-Takayanagi formula does fail in general, or there is no dual for BTZ bubble. We assumed Ryu-Takayanagi formula to get boundary entanglement entropy.}

The light-cone cuts method on the other hand, succeeded to reconstruct the conformal metrics of both BTZ and BTZ bubble.
They first used bulk null geodesics to obtain light-cone cuts.
%
% and the light-cone cuts method needs null geodesics.
%Both geodesics can be analytically calculated in BTZ black hole and BTZ bubble space.
%From the spacelike geodesics, the differential entropy \cite{Balasubramanian:2013lsa} can be calculated, because the Ryu-Takayanagi surfaces are lines in $(2+1)$ dimensional spacetime.
%However, the reconstruction by the hole-ography method is not completed in this paper.
%On the other hand, they calculated cuts from null geodesics, and succeeded in reconstructing the bulk conformal metrics in the both spacetimes.




%%
\bibliographystyle{jhep} 
\bibliography{reference.bib}
\end{document}
