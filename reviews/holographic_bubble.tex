\documentclass[12pt]{article}

\usepackage{paper}

%commands
\renewcommand{\thefootnote}{\fnsymbol{footnote}}
\renewcommand{\theequation}{\arabic{equation}}

\date{}

\begin{document}
{\Large\noindent
[Brief Review\footnote{
The reviewer: Daichi Takeda (takedai.gauge@gmail.com)
}]\\[2mm]
%title
\textbf{Holographic Reconstruction of Bubbles\cite{Burda:2018rpb}
}
}

\noindent
\hfill
\textbf{Philipp Burda, Ruth Gregory, Akash Jain}%author

\renewcommand{\thefootnote}{\arabic{footnote})}
\setcounter{footnote}{0}
\vspace{12pt}
%%body
They examined whether the two bulk reconstruction methods, hole-ography
\cite{Balasubramanian:2013lsa,Czech:2014ppa}, and the light-cone cuts method \cite{Engelhardt:2016wgb}, 
work for BTZ black hole and BTZ bubble (explained below).
Since we do not know the dual theories of the two geometries well,
the knowledge of the bulk geometries is assumed at first, then forgot.
Entanglement entropy is computed by using Ryu-Takayanagi formula, and light-cone cuts are found from null geodesics.
We should note that the former uses a holographic interpretation, but the latter do not.

BTZ black hole without charge and angular momentum is a $(2+1)$ dimensional black hole spacetime which asymptotes to $\mathrm{AdS}_3$:
\begin{align}
	\d s^2 = \frac{1}{z^2}\l[-(1-Mz^2)\d t^2 + \d x^2 + \frac{\ell^2}{1-Mz^2}\d z^2\r].
\end{align}
BTZ bubble metric is given by
\begin{align}
	\d s^2 = 
	\l\{
		\begin{array}{cc}
			\displaystyle \frac{1}{z^2}\l[-(1-M_+z^2)\d t^2 + \d x^2 + \frac{\ell_+^2}{1-M_+z^2}\d z^2\r] & \mathrm{for}~z\le Z(\tau),\\[5mm]
			\displaystyle\frac{1}{z^2}\l[-(1-M_-z^2)\d t^2 + \d x^2 + \frac{\ell_-^2}{1-M_-z^2}\d z^2\r] & \mathrm{for}~z\ge Z(\tau),
		\end{array}
	\r.
\end{align}
where $z = Z(\tau)$ is  a timelike hypersurface with timelike coordinate $\tau$, and called bubble.
From some suitable conditions, they require
\begin{align}
	M_+ > M_-,\qquad \frac{\ell_+}{M_+} \geq \frac{\ell_-}{M_-}.
\end{align}
There is a thin bubble having tension on $z = Z(\tau)$.

It turned out that this conjecture does not hold for BTZ bubble, while it holds for BTZ.
However, considering that BTZ bubble is an unstable geometry, we cannot be certain if
BTZ bubble has a holographic dual and Ryu-Takayanagi formula holds for it.

On the other hand, the light-cone cuts method succeeded to reconstruct the conformal metrics of both BTZ and BTZ bubble,
but we should note that no holographic element comes into their calculations and results about the light-cone cuts method.




%%
\bibliographystyle{jhep} 
\bibliography{reference.bib}
\end{document}
