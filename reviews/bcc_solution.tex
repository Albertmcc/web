\documentclass[12pt]{article}

\usepackage{paper}

%commands
\renewcommand{\thefootnote}{\fnsymbol{footnote}}
\renewcommand{\theequation}{\arabic{equation}}

\date{}

\begin{document}
{\Large\noindent
[Brief Review\footnote{
The reviewer: Daichi Takeda (takedai.gauge@gmail.com)
}]\\[2mm]
%title
\textbf{Solutions from boundary condition changing operators in open string field theory\cite{Kiermaier:2010cf}
}
}

\noindent
\hfill
\textbf{M. Kiermaier, Y. Okawa and P. Soler}%author

\vspace{12pt}
%%body
The general form of analytic solution for regular marginal deformations is introduced by using the boundary condition changing (bcc) operators.
The bcc operators $\sigma_L$ and $\sigma_R$ satisfy
\begin{align}
	\sigma_L(a)\sigma_R(b) = \exp\l[\lambda\int_a^b\d t\,V(t) \r],
\end{align}
where $V(t)$ is the primary operator with weight $1$, causing the marginal deformation.
The bcc operators are assumed to be primary operators with weight $0$.
The solution shown in this paper is given as follows:
\begin{align}
	\Psi = -\frac{1}{\sqrt{1-K}}(Q_\mathrm{B}\sigma_L)
	\l[
	\sigma_R + \frac{B}{1-K}Q_\mathrm{B}\sigma_R
	\r]
	\frac{1}{\sqrt{1-K}}.\label{eq:bcc_solution}
\end{align}
It can be confirmed that this is a pure-gauge solution.

The BPZ inner product between the $\Psi$ in \eqref{eq:bcc_solution} and a generic state $\phi = -c\partial c\phi_m$ with $\phi_m$ being a matter primary operator of weight $0$, $\braket{\phi,\Psi}$ is important especially for the rolling tachyon.
Because of the introduction of the bcc operator, this can be calculated by the factorization into the three-point correlation function $\braket{f\circ \phi_m(0)\sigma_L(a)\sigma_R(b)}$ and the ghost part with ghost number $3$.
Both can be calculated because the three-point correlator of primary fields is fixed  in the 2-dim CFT, and the latter is usual calculations seen in the \textit{KBc} subalgebra.
The result has the form
\begin{align}
	\braket{\phi,\Psi} = g(h) \braket{\phi_m(0)\sigma_L(1)\sigma_R(\infty)}_{\mathrm{matter,\,UHP}}
\end{align}
with $h$ being the weight of $\phi$, where $g(h)$ does not depend on $V$.
This is applied for the rolling tachyon in section 4.



%
\bibliographystyle{jhep} 
\bibliography{reference.bib}
\end{document}
