\documentclass[12pt]{article}

\usepackage{paper}

%commands
\renewcommand{\thefootnote}{\fnsymbol{footnote}}
\renewcommand{\theequation}{\arabic{equation}}

\date{}

\begin{document}
{\Large\noindent
[Brief Review\footnote{
The reviewer: Daichi Takeda (takedai.gauge@gmail.com)
}]\\[2mm]
%title
\textbf{Nuts and Bolts for Creating Space\cite{Czech:2014ppa}
}
}

\noindent
\hfill
\textbf{Bartlomiej Czech, Lampros Lamprou}%author

\renewcommand{\thefootnote}{\arabic{footnote})}
\setcounter{footnote}{0}
\vspace{12pt}
%%body

They introduced the method to reconstruct the time slice surface of $\mathrm{AdS}_3$ from the differential entropy of $\mathrm{CFT}_2$ \cite{Balasubramanian:2013lsa} (see \href{https://albertmcc.github.io/web/reviews/Hole_ographic_spacetime.pdf}{\color{blue}my review}).
Specifically they found the method to detect the bulk points and to measure the distances between two points on the same time slice.
In their method, the distances between any timelike separated two points cannot be measured.

\noindent
\textbf{The detection of points}

Let $\alpha(\theta)$ be a boundary function introduced in \cite{Balasubramanian:2013lsa}, which is also seen in \href{https://albertmcc.github.io/web/reviews/Hole_ographic_spacetime.pdf}{\color{blue}my review}.\footnote{The notation adopted here is checked in  my review.}
Any bulk closed curve $R = R(\phi)$ corresponds to a certain boundary function $\alpha(\theta)$.
In the limit of the bulk curve shrinking, we get a point which is a center of the infinitesimal closed curve.

They interpreted this shrinking limit to $\alpha(\theta)$ to find the way to detect bulk points from the boundary as follows.
First, note that the Gauss-Bonnet theorem states that the following relation holds in negatively curved space:
\begin{align}
	\oint_{\partial A}\d\tau \sqrt{h}K = 2\pi - \int_A \d A\,R\ge 2\pi.\label{eq:Gauss}
\end{align}
Here $\partial A$ means the closed curve $R = R(\phi)$, $A$ is the area enclosed by the curve, $\d\tau\sqrt{h}$ is the line element along the curve, $K$ is the extrinsic curvature, and $\d A$ is the areal element.
In the shrinking limit, we obtain
\begin{align}
	\lim_{A\to \mathrm{point}} ~\oint_{\partial A}\d\tau \sqrt{h}K = 2\pi.
\end{align}
In addition, the integrand of l.h.s.\ in \eqref{eq:Gauss} can be expressed by $\alpha(\theta)$ as
\begin{align}
	\d\tau \sqrt{h} K = \frac{\d\theta\sqrt{1-\alpha'(\theta)^2}}{\sin\alpha(\theta)}.
\end{align}

From these facts, the boundary function $\alpha(\theta)$ corresponds to a certain bulk point if and only if it extremizes the functional
\begin{align}
	I[\alpha] = \int_0^{2\pi}\d\theta \frac{\sqrt{1-\alpha'(\theta)^2}}{\sin\alpha(\theta)},\label{eq:action}
\end{align}
the Eular-Lagrange equation of which is a second order differential equation.
Therefore, $\alpha(\theta)$ has two constants, which are necessary and sufficient to identify bulk points on a time slice.
Noting the relation
\begin{align}
	\frac{1}{\sin^2\alpha} = -\frac{2G}{L}\frac{\d^2S(\alpha)}{\d\alpha^2},
\end{align}
the action \eqref{eq:action} can be rewritten as 
\begin{align}
	I[\alpha] = \int_0^{2\pi}\d\theta\,\sqrt{-\l.\frac{\d^2S(\alpha)}{\d\alpha^2}\r|_{\alpha = \alpha(\theta)}(1-\alpha'(\theta)^2)}~~.\label{eq:aaction}
\end{align}
They conjectured that this action also holds for any two-dimensional boundary theory dual to an asymptotically $\mathrm{AdS}_3$.
In other words, the collection of functions $\alpha(\theta)$ extremizing \eqref{eq:aaction} is the copy of the bulk points.
The elements of the collection are called ``point function".
A counterexample is discussed in \cite{Burda:2018rpb} (see \href{https://albertmcc.github.io/web/reviews/holographic_bubble.pdf}{\color{blue}my review}).


\noindent
\textbf{The measurement of distances}

Next, let us introduce the notion of the distance which coincides with the bulk one, in the set of point functions $\mathcal P$.
Now the differential entropy $E[\alpha]$ is used to define the distance.

Let $\gamma_{AB}(\theta) := \min\{\alpha_A(\theta),\alpha_B(\theta)\}$ for all $\theta$ with $\alpha_A,\alpha_B\in\mathcal P$ ($\gamma$ is defined as a pointwise minimum among $\alpha_A$ and $\alpha_B$).
The functional $d(A,B)$ defined as follows can be regarded as the distance between $A$ and $B$ measured in the units of $4G$:
\begin{align}
	d(A,B) := \frac{1}{2}E[\gamma].\label{eq:dist}
\end{align}
It is shown that this functional satisfies the axioms of a distance function.\footnote{
Note that this is the distance defined on a time slice, so the distance is always positive.}

In the remaining, we see why $d(A,B)$ can be regarded as the distance on the time slice.
Let us assume that a generic boundary function $\alpha(\theta)$ has a point $\theta_k$ at which $\alpha'(\theta)$ jumps.
From the corollary shown in \cite{Balasubramanian:2013lsa},\footnote{
See the end of \href{https://albertmcc.github.io/web/reviews/Hole_ographic_spacetime.pdf}{\color{blue}my review}}
we get the relation
\begin{align}
	E[\alpha] &= \frac{1}{2}\int_{0}^{\theta_k-0}\d\theta\,\l.\frac{\d S(\alpha)}{\d \alpha}\r|_{\alpha = \alpha(\theta)} + \frac{1}{2}\int_{\theta_k+0}^{2\pi}\d\theta\,\l.\frac{\d S(\alpha)}{\d \alpha}\r|_{\alpha = \alpha(\theta)}\nonumber\\[2mm]
	&= \frac{(\mbox{left length}) + (\mbox{right length}) + (\mbox{geodesic length in } \phi(\theta_k-0)\leq \phi\leq \phi(\theta_k+0))}{4G}.
\end{align}
This means that the bulk curve $(R,\phi) = (R(\theta),\phi(\theta))$ corresponding to $\alpha(\theta)$ has a jump at $\theta$, and the relation
\begin{align}
	E[\alpha] = \frac{\mbox{length}}{4G}
\end{align}
still holds if the jump is smoothly complemented by the geodesic.

Let $\gamma(\theta) = \min\{\alpha(\theta),\beta(\theta)\}$ with $\alpha$ and $\beta$ smooth.
Generally, $\gamma$ has points at which $\gamma'(\theta)$ jumps, and these points correspond to the points $\gamma$ changes the value from $\alpha$ to $\beta$ or vice versa.
Therefore, for each $\theta_0$ giving jumps, there must exist the geodesic which smoothly connects $(R_\alpha(\theta_0),\phi_\alpha(\theta_0))$ and $(R_\beta(\theta_0),\phi_\beta(\theta_0))$, where $(R_\alpha(\theta),\phi_\alpha(\theta))$ is a curve corresponding to $\alpha$ and $(R_\beta(\theta),\phi_\beta(\theta))$ to $\beta$.
This means that there is a convex curve which covers $(R_\alpha(\theta),\phi_\alpha(\theta))$ and $(R_\beta(\theta),\phi_\beta(\theta))$ (see Figure 6 in \cite{Czech:2014ppa}).
In the shrinking limit of $(R_\alpha(\theta),\phi_\alpha(\theta))\to A$ and $(R_\beta(\theta),\phi_\beta(\theta))\to B$, the convex curve becomes the identical two geodesic going from $A$ to $B$, by which we can verify \eqref{eq:dist}.

The applications to conical $\mathrm{AdS}_3$ and BTZ are also discussed in the paper.












%%
\bibliographystyle{jhep} 
\bibliography{reference.bib}
\end{document}
