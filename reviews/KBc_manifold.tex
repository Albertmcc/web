\documentclass[12pt]{article}

\usepackage{paper}

%commands
\renewcommand{\thefootnote}{\fnsymbol{footnote}}

%title,author
\title{\textbf{Review}\\
Interior Product, Lie derivative and Wilson Line in the \textit{KBc} Subsector of Open String Field Theory
}
\author{Hiroyuki Hata and Daichi Takeda}
\date{}
\begin{document}
{\Large\noindent
[Brief Review\footnote{
The reviewer: Daichi Takeda (takedai.gauge@gmail.com)
}]\\[2mm]
\textbf{Interior Product, Lie derivative and Wilson Line in the \textit{KBc} Subsector of Open String Field Theory
}
}

\noindent
\textbf{Author: Hiroyuki Hata and Daichi Takeda}

\vspace{12pt}
%%body
\textit{KBc} algebra has been described classical solutions like tachyon condensation and multi-$\mathrm{D}_{25}$ branes.
These solutions have singularities caused by $K\sim 0$, but reproduce the exact energies expected by string theory.

We addressed how to emerge Chan-Paton factors around the multi-$\mathrm{D}_{25}$ brane solution found by Murata and Schnabl \cite{Murata:2011ex}.
Comparing with the action with Chan-Paton factors, the fluctuation $\Delta\Psi$ around the $N$-brane solution is expected to expanded by the modified form of vertex operators $V_a^\ddag\mathcal O_F(k) V_b$  with some operators $V_a\,(a = 1,\cdots,\,N)$ carrying no ghost.
Then we find that the question becomes whether there exist $N$ operators $V_a$ satisfying
\begin{align}
	V_a(\overleftarrow Q_\mathrm{B}+\Psi_0) = 0\quad
	\mathrm{and}\quad
	V_aV_b^\ddag = \delta_{ab},\label{eq:Va_coditions}
\end{align}
where $\Psi_0$ is the $N$-brane solution and $A\overleftarrow Q_\mathrm{B} = -(-1)^{|A|}Q_\mathrm{B}A$.
The similar equation for the former can be seen in Wilson lines of gauge theories, and the formal resemblance between SFT and Chern-Simons theory has been known,
so we are driven to find the correspondence of Wilson line in SFT.

In the process to find this, we established the \textit{KBc} manifold, which includes infinite sets of distinct $(K,B,c)$, of course satisfying \textit{KBc} algebra, as its points.
On the manifold, we succeeded in finding the interior product, Lie derivative and Lie bracket.
The Lie derivative gives infinitesimal deformations in the space of solutions.
By using the interior product, we finally obtained a candidate for the SFT Wilson line.
However, the task to solve \eqref{eq:Va_coditions} is still left, and the information about \textit{KBc} manifold may be not enough.

%%%%%
\bibliographystyle{jhep} 
\bibliography{reference.bib}
\end{document}
