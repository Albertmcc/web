\documentclass[12pt]{article}

\usepackage{paper}

%commands
\renewcommand{\thefootnote}{\fnsymbol{footnote}}
\renewcommand{\theequation}{\arabic{equation}}

\date{}

\begin{document}
{\Large\noindent
[Brief Review\footnote{
The reviewer: Daichi Takeda (takedai.gauge@gmail.com)
}]
\hfill{\normalsize last updated: \today}
\\[2mm]
%title
\textbf{Bulk reconstruction of metrics with a compact space asymptotically \cite{Hernandez-Cuenca:2020ppu}
}
}

\noindent
\hfill
\textbf{S. Hern'andez-Cuenca and G.T. Horowitz}%author

\vspace{12pt}
%%body
The light-cone cut method \cite{Engelhardt:2016wgb} 
(see \href{https://albertmcc.github.io/web/reviews/light-cone1.pdf}{\color{blue}my review}) 
is extended to the spacetime with a compact space.
Here again we focus on the past cuts.

In order to reconstruct the bulk conformal metric by light-cone cuts, we have to collect inward null vectors at points on cuts,
the component of the compact space being taken into account.
Since the compact space shrinks up, the coordinate on $\mathbb S^k$ of $x \in C^-(p)$ 
is defined as asymptotical value of null geodesics, which is denoted by $\Phi(x)$.
Thus, the point would have two ore more null geodesics coming into it,
although it is not a caustic.
Therefore, we define the regular point on $C^-(p)$ as the point into which a unique null geodesic comes, 
in order to get well-defined asymptotical coordinate on $\mathbb S^k$.
The set of regular points on $C^-(p)$ is written as $G^-(p)$.
Then we define the extended light-cone past cuts as follows:
\begin{align}
	\mathcal C^-(p) = \{(x,\Phi(x))|\,x\in G^-(p)\}.
\end{align}
The holographic procedure to find $\Phi(x)$ is explained in the paper.

The similar proof to the one in \cite{Engelhardt:2016wgb} gives the fact that, 
when $\mathcal C^-(p)$ and $\mathcal C^-(q)$ intersect at exactly one point, then $p$ and $q$ are null related.
Following the same process in the previous paper gives the light-cone of $P (\leftrightarrow C^-(p))$ 
in the space of extended past cuts, and determines the conformal metric at $P$, that is, the conformal metric at $p$.

In this paper, the spacetime is assumed to be asymptotically $\mathrm{AdS}_n\times\mathbb S^k$,
but  the asymptotical form of the compact space would be generalized (discussed in section 5).



%%
\bibliographystyle{jhep} 
\bibliography{reference.bib}
\end{document}
