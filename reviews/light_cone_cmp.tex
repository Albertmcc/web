\documentclass[12pt]{article}

\usepackage{paper}

%commands
\renewcommand{\thefootnote}{\fnsymbol{footnote}}
\renewcommand{\theequation}{\arabic{equation}}

\date{}

\begin{document}
{\Large\noindent
[Brief Review\footnote{
The reviewer: Daichi Takeda (takedai.gauge@gmail.com)
}]\\[2mm]
%title
\textbf{Bulk reconstruction of metrics with a compact space asymptotically \cite{Hernandez-Cuenca:2020ppu}
}
}

\noindent
\hfill
\textbf{S. Hern'andez-Cuenca and G.T. Horowitz}%author

\vspace{12pt}
%%body
The light-cone cut method \cite{Engelhardt:2016wgb,Engelhardt:2016crc} is extended to the spacetime with a compact space.
Here again we focus on the past cuts.

In order to reconstruct the bulk conformal metric by light-cone cuts, one have to collect inward null vectors at points on cuts, including the component of the compact space.
Because the compact space shrinks up, the coordinate on $\mathbb S^k$ of $x \in C^-(p)$ is defined as asymptotical value of null geodesics, which is denoted by $\Phi(x)$.
When the spacetime has compact space, although a point on a cut is not a caustic, the point would have two ore more null geodesics coming into the point.
Therefore, we define the regular point on $C^-(p)$ as the point into which a unique null geodesic comes, in order to get well-defined asymptotical coordinate on $\mathbb S^k$.
The set of regular points of $C^-(p)$ is written as $G^-(p)$.
Then we define the past extended light-cone cuts as follows:
\begin{align}
	\mathcal C^-(p) = \{(x,\Phi(x))|\,x\in G^-(p)\}.
\end{align}

The similar proof to the one in \cite{Engelhardt:2016wgb} gives the fact that, when $\mathcal C^-(p)$ and $\mathcal C^-(q)$ intersect at exactly one point, then $p$ and $q$ are null related.
Following the same process in the previous paper gives the light-cone of $P (\leftrightarrow C^-(p))$ in the space of past extended cuts, and determines the conformal metric at $P$, that is, the conformal metric at $p$.

In this paper, the spacetime is assumed to be asymptotically $\mathrm{AdS}_n\times\mathbb S^k$, but  the asymptotical form of the compact space would be generalized (discussed in section 5).



%%
\bibliographystyle{jhep} 
\bibliography{reference.bib}
\end{document}
