\documentclass[12pt]{article}

\usepackage{paper}

%commands
\renewcommand{\thefootnote}{\fnsymbol{footnote}}
\renewcommand{\theequation}{\arabic{equation}}

\date{}

\begin{document}
{\Large\noindent
[Brief Review\footnote{
The reviewer: Daichi Takeda (takedai.gauge@gmail.com)
}]\\[2mm]
%title
\textbf{Conformal Rigidity from Focusing\cite{Folkestad:2021kyz}
}
}

\noindent
\hfill
\textbf{$\mathring{\textrm{A}}\textrm{smund}$ Folkestad, Sergio Hern\' andez-Cuenca}%author

\renewcommand{\thefootnote}{\arabic{footnote})}
\setcounter{footnote}{0}
\vspace{12pt}
%%body
We focus on the byproduct of the paper related to the light-cone cuts method \cite{Engelhardt:2016wgb,Engelhardt:2016crc,Hernandez-Cuenca:2020ppu} in this paper.

The null curvature condition (NCC) is that $R_{\mu\nu}k^\mu k^\nu \geq 0$ for any null vector $k^\mu$, 
which ensures with Raychaudhuri equation that null geodesic congruences are focused.
Especially, $R_{\mu\nu}k^\mu k^\nu = 0$ (for any null vector $k^\nu$) is called zero null curvature condition.
NCC is equivalent to the null energy condition $T_{\mu\nu}k^\mu k^\nu \geq 0$ (for any null vector $k^\nu$), 
when Einstein equation is assumed.
Hawking's area theorem and various other theorems are shown under NCC, 
and NCC are supposed to be physically natural because of the attractive nature of gravity.
It is also well motivated to assume NCC, because of some facts (see the first paragraph in page 3).

The next theorem would be useful in the light-cone cuts method, 
in order to determine the conformal factor of the bulk metric, which cannot be identified by the method.
\\[12pt]
\textbf{Theorem 4}\footnote{
The theorem number is the same one in the paper.
}~~
Let $(M,g)$ be an asymptotically locally AdS spacetime with zero null curvature and $\Omega(x)~(x\in M)$ be a positive
 function extending to $\partial M$ (the conformal boundary) with the same value everywhere.
Let $A$ be the subset of $M$, of which any point $p$ lies on at least one complete null geodesic with both endpoints on 
$\partial M$.
Then at least one of the following is true.
\begin{enumerate}
	\item The spacetime $(g,\Omega^2M)$ violates the NCC on $A$.
	\item The Weyl factor $\Omega$ is constant on $A$
	\item The spacetime $(M,\Omega^2g)$ is not asymptotically locally AdS.
\end{enumerate}

Let us suppose that we get the conformal metric $\Omega^2 g \sim g$ by the light-cone cuts method.
If the conformal factor $\Omega_0$ such that $g_{\Omega_0} = \Omega_0^2 g$ gives vanishing null curvature exists, 
then the spacetime $(M,g_{\Omega_0})$ satisfies the assumption of the theorem 4.
The area $A$ is (naively?) equal to $J^+(\partial M)\cap J^-(\partial M)$ here.
The exact metric can be obtained by multiplying $g_{\Omega_0}$ by a certain conformal factor $\Omega_\ast$.
Assuming that $\Omega_\ast$ is in the class of the conformal factors which has the asymptotic behavior written in the theorem,
 and accepting the claim that the metric predicted by the boundary QFT should satisfy NCC, 
 the Weyl factor $\Omega_\ast$ is constant on $A$.\footnote{
The possibility of No.3 is omitted in advance because we are discussing the problem of AdS/CFT correspondence.
}
This means that $g_{\Omega_\ast}$ is the bulk metric, up to the constant rescaling of the metric.

Finally, it should be noted that pure gravity solutions of Einstein equation always satisfy zero null curvature condition.

%%
\bibliographystyle{jhep} 
\bibliography{reference.bib}
\end{document}
