\documentclass[12pt]{article}

\usepackage{paper}

%commands
\renewcommand{\thefootnote}{\fnsymbol{footnote}}
\renewcommand{\theequation}{\arabic{equation}}
\newcommand{\AdS}[0]{\textrm{AdS}}
\newcommand{\CFT}[0]{\textrm{CFT}}

\date{}

\begin{document}
{\Large\noindent
[Brief Review\footnote{
The reviewer: Daichi Takeda (takedai.gauge@gmail.com)
}]\\[2mm]
%title
\textbf{A hole-ographic spacetime
\cite{Balasubramanian:2013lsa}}
}

\noindent
\hfill
\textbf{Vijay Balasubramanian, Borun D. Chowdhury,}

\hfill
\textbf{Bartlomiej Czech, Jan de Boer, Michal P. Heller.}
%author

\renewcommand{\thefootnote}{\arabic{footnote})}
\setcounter{footnote}{0}
\vspace{12pt}
%%body

In pure $\AdS_3/ \CFT_2$ correspondence, 
they proposed a boundary quantity called differential entropy, which measures the circumference of a bulk curve.

The static coordinate of $\AdS_3$ is
\begin{align}
	\d s^2 = -\l(1+\frac{R^2}{L^2}\r)\d T^2 + \l(1 + \frac{R^2}{L^2}\r)^{-1}\d R^2 + R^2 \d \phi^2.
\end{align}
The vacuum entanglement entropy for any interval with its length $2\alpha$, is computed as
\begin{align}
	S(\alpha) = \frac{c}{3}\ln\l(\frac{2L}{\mu}\sin\alpha\r).
\end{align}

In the paper, they introduced \textit{differential entropy} as the modified version of the entanglement entropy, 
in which the UV regulator $\mu$ is removed.

\noindent
\textbf{Definition}~~
Let $\alpha(\theta)$ be a smooth function of $\theta$ ($\alpha(\theta+2\pi) = \alpha(\theta)$) 
defined on a boundary time slice $T=0$.
The differential entropy, which is a functional of $\alpha(\theta)$, is defined as
\begin{align}
	E[\alpha] := \frac{1}{2}\int_0^{2\pi}\d\theta\,\l.\frac{\d S(\alpha)}{\d \alpha}\r|_{\alpha = \alpha(\theta)}.
\end{align}

\noindent
In this holographic formula, function $\alpha$ corresponds a certain closed bulk curve on $T=0$.
This is understood as follows.
There is a unique spacelike geodesic $\gamma_\alpha(\theta)$ anchored at $\theta \pm \alpha(\theta)$ on the boundary.
If we move $\theta$, $\gamma_\alpha(\theta)$ sweeps out the slice $T=0$, leaving a hole on the slice.
The hole is the one we are considering and each $\gamma_\alpha(\theta)$ is tangent on the hole.

In the remaining of this note, we overview the derivation of the formula.
First, let us see how $R(\phi)$ is related to $\alpha(\theta)$.
For each point $\phi$ on $R=R(\phi)$, there is a unique geodesic on $T=0$ which is tangent to the curve at $\phi$.
Each of such spacelike geodesics parametrized by $\phi$ reaches the boundary points $\theta = \theta(\phi) \pm \alpha(\phi)$:
\begin{align}
	\alpha(\phi) &= \arctan\l[\frac{L}{R}\,\sqrt{1 + \frac{L^2}{R^2+L^2}\l(\frac{\d \ln R(\phi)}{\d\phi} \r)^2} \,\,\r],\\
	\theta(\phi) &= \phi - \arctan\l[\frac{L^2}{R^2 + L^2}\frac{\d\ln R(\phi)}{\d \phi} \r].
\end{align}
These two equations relate the curve $R=R(\phi)$ to the interval function $\alpha(\theta)$.
These formula are inverted to be
\begin{align}
	R(\theta) &= L\cot\alpha(\theta) \sqrt{\frac{1 + \alpha\p(\theta)^2\tan^2\alpha(\theta)}{1-\alpha\p(\theta)^2}},\label{eq:R}\\
	\phi(\theta) &= \theta - \alpha\p(\theta)\tan\alpha(\theta).\label{eq:phi}
\end{align}

Next, let us see the formula
\begin{align}
	E[\alpha] = \frac{\mathrm{circumference~of~}(R,\phi) = (R(\theta),\phi(\theta))}{4G}.\label{eq:conj}
\end{align}
To see this, we rewrite the right hand side by the bulk terms as
\begin{align}
	\mathrm{r.h.s.} = \int_0^{2\pi} \frac{\d\phi}{4G}\sqrt{\l(1 + \frac{R^2}{L^2}\r)^{-1}\l(\frac{\d R}{\d \phi}\r)^2 + R^2}.
\end{align}
Note that we should regard $\phi$ and $R(\phi)$ in this expression as $\phi(\theta)$ and $R(\theta)$ given by \eqref{eq:R} 
and \eqref{eq:phi}.
Comparing each integrand of both hand sides in \eqref{eq:conj}, we get
\begin{align}
	\mathrm{l.h.s.} &\rightarrow \frac{\d\theta}{2}\l.\frac{\d S(\alpha)}{\d\alpha}\r|_{\alpha = \alpha(\theta)},\\
	\mathrm{r.h.s.} &\rightarrow \frac{\d\phi}{4G}\sqrt{\l(1 + \frac{R^2}{L^2}\r)^{-1}\l(\frac{\d R}{\d \phi}\r)^2 + R^2}.
\end{align}
Subtracting the integrand of the l.h.s.\ from the r.h.s., we find the result is $\d f$, with $f$ defined as
\begin{align}
	f(\theta) := \l[\frac{L}{8G}\ln\l(\frac{\sin(\alpha(\theta)+\phi(\theta)-\theta)}{\sin(\alpha(\theta)-\phi(\theta) + \theta)} \r) \r].
\end{align}
Therefore the formula \eqref{eq:conj} is shown, since $\d f$ is an exact form.
It is also shown that $f(\theta)$ is equal to the proper length along the spacelike geodesic on $T=0$ between two points $\phi$ and $\theta(\phi)$ of $R = R(\phi)$.
Then we get a corollary: let $L$ be a proper length of the bulk curve $\{(R,\phi) = (R(\theta),\phi(\theta))|\theta_i\le\theta\le \theta_f\}$, then
\begin{align}
	\frac{L}{4G} = \frac{1}{2}\int_{\theta_i}^{\theta_f}\d\theta\,\l.\frac{\d S(\alpha)}{\d \alpha}\r|_{\alpha = \alpha(\theta)}  + f(\theta_f) - f(\theta_i).
\end{align}

A heuristic approach is taken in their paper, and a brief but more concrete review appears in \cite{Czech:2014ppa}.


































%%
\bibliographystyle{jhep} 
\bibliography{reference.bib}
\end{document}
