\documentclass[12pt]{article}

\usepackage{paper}

%commands
\renewcommand{\thefootnote}{\fnsymbol{footnote}}
\renewcommand{\theequation}{\arabic{equation}}
\date{}

\begin{document}
{\Large\noindent
\footnote{
Written by Daichi Takeda (takedai.gauge@gmail.com)
}]
\hfill{\normalsize last updated: \today}
\\[2mm]
%title
\textbf{Towards a Reconstruction of General Bulk Metrics\cite{Engelhardt:2016wgb}
}
}

\noindent
\hfill
\textbf{N. Engelhardt and G.T. Horowitz}%author

\vspace{12pt}
%%body
A new approach for the bulk reconstruction, the method using the light-cone cuts, is introduced.
The past (future) light-cone cut of the bulk point $p$ is defined as
\begin{align}
	C^-(p) = \partial J^-(p)\cap \partial M~(C^+(p) = \partial J^+(p)\cap \partial M),
\end{align}
where $M$ is the spacetime assumed to be asymptotically $\mathrm{AdS}_n$ with some natural conditions 
for mathematical reasons.
From now on, we focus on past cuts, but the same discussion can be applied to future cuts.

First, using the ``bulk-point singularity" \cite{Maldacena:2015iua}, we can get the past cuts of the points 
in $J^+(\partial M)\cap J^-(\partial M)$.
A certain class of divergence of boundary to boundary massless correlators,
is caused by the vertex being in the bulk at which total null momenta is conserved.
Such a divergence is called bulk-point singularity.
Let us consider a boundary correlator consisting of $n$ points $x_1,\cdots,x_n$ which are spacelike separated each other, 
and two points $z_1, z_2$ in the past of $x_i$:
\begin{align}
	\langle\mathcal O(z_1)\mathcal O(z_2)\mathcal O(x_1)\cdots\mathcal O(x_n)\rangle.
\end{align}
If we move $z_1$ and $z_2$ as the correlator diverges while keeping the momentum conservation,
we see that $z_1$ and $z_2$ draw the past cut of bulk point null-separated from $x_i$.

Repeating the above process, we obtain the set of past cuts.
However, since we do not know about the bulk geometry in the reconstruction process,
we cannot know which bulk points corresponds to which past cuts.
In principle, however, past cuts can be labeled by some $n$ parameters related to $x_i$.
Then, let us use $\lambda = (\lambda^1,\cdots,\lambda^n)$ to denote past cuts as $C^-_\lambda$.
The holographic interpretation is that $\lambda$ is a bulk point described in some coordinate,
and we write the set of past cuts as $\mathcal M^-$.
Our remaining task is to introduce a metric to $\mathcal M^-$ properly.

It is shown in the appendix that, if $C^-(p)$ and $C^-(q)$ are tangent each other 
precisely at one point, then $p$ and $q$ are null related.
Thus, we should introduce the metric of $\mathcal M^-$ so that if $C^-_{\lambda}$ and $C^-_{\lambda'}$
is tangent at one point, then $\lambda$ and $\lambda'$ is null-separated.
This determines the causal structure of $\mathcal M^-$.

We can compute the conformal metric (the metric up to conformal factors) from the above property.
The null generators at $\lambda\in \mathcal M^-$ are obtained by looking for other past cuts tangent to $C^-_{\lambda}$.
Let $\sigma = (\sigma^1,\cdots,\sigma^{n-2})$ be the parameter describing the points on 
$C^-_{\lambda}$ as $C^-_{\lambda}(\sigma) \in \partial M$.
The condition which $C^-_{\lambda_2}$ is tangent to $C^-_{\lambda_1}$ is as follows:
\begin{align}
	\exists\sigma_1,~\exists\sigma_2,\quad C^-_{\lambda_1}(\sigma_1) = C^-_{\lambda_2}(\sigma_2)\quad\mathrm{and}\quad \l.\nabla_{\sigma} C^-_{\lambda_1}(\sigma)\r|_{\sigma = \sigma_1} = \l.\nabla_{\sigma} C^-_{\lambda_2}(\sigma) \r|_{\sigma = \sigma_2}.
\end{align}
By solving the conditions for $\lambda^1 = \lambda$ and $\lambda^2 = \lambda + \delta\lambda$ to 
$\mathcal O(\delta\lambda)$, we obtain $n(n+1)/2$ vectors at $\lambda$.
Since the vectors must be null, the conformal metric is uniquely determined.


%%
\bibliographystyle{jhep} 
\bibliography{reference.bib}
\end{document}
