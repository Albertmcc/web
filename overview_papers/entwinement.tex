\documentclass[12pt]{article}

\usepackage{paper}

%commands
\renewcommand{\thefootnote}{\fnsymbol{footnote}}
\renewcommand{\theequation}{\arabic{equation}}
\newcommand{\overview}[1]{\href{#1}{\color{blue}my overview}}
\newcommand{\conical}[0]{\mathrm{AdS}_3/\mathbb Z_n}

\date{}

\begin{document}
{\Large\noindent
[Overview\footnote{
Written by Daichi Takeda (takedai.gauge@gmail.com)
}]
\hfill{\normalsize last updated: \today}
\\[2mm]
%title
\textbf{Entwinement and the emergence of spacetime\cite{Balasubramanian:2014sra}
}
}

\noindent
\hfill
\textbf{Vijay Balasubramanian, Borun D. Chowdhury, Bartlomiej Czech, Jan de Boer}%author

\renewcommand{\thefootnote}{\arabic{footnote})}
\setcounter{footnote}{0}
\vspace{12pt}
%%body
\noindent
In the bulk reconstruction program, it is known that there is a bulk region called entanglement shadow, which is a region minimal surfaces cannot enter.
This is problematic because Ryu-Takayanagi formula, which is known as a key formula in reconstructing spacetimes, cannot be used to construct the metric in an entanglement shadow.
In the paper, they proposed another notion called \textit{entwinement} as a boundary quantity to go deeper into the bulk.
Recalling that the radial coordinate $r$ in the bulk is interpreted as the energy scale of the boundary, we need a boundary quantity characterizing smaller energy scale to know the inside of an entanglement shadow.
The entwinement reflects the entanglement entropy between ``internal" degrees of freedom, while the ordinary spatial entanglement entropy does not.

\subsection*{The setup}
As an example, they considered the AdS$_3$ spacetime with a conical defect:
\begin{align}
	\d s^2 = -\l(\frac{1}{n^2}+\frac{r^2}{L^2}\r)\d t^2 + \l(\frac{1}{n^2}+\frac{r^2}{L^2}\r)\d r^2 + r^2\d\theta^2,
	\qquad n\in \mathbb N.\label{eq: conical AdS}
\end{align}
Here $\theta$ and $\theta + 2\pi$ are identified.
Since this geometry can be obtained by the identification $\theta\sim \theta + 2\pi/n$ in the pure AdS$_3$, it is denoted as $\conical$.
The $\mathbb Z_n$-identification can be regarded as a discrete gauge symmetry, hence the corresponding boundary theory is in a $\mathbb Z_n$-symmetric state.
Thus, the boundary theory has the internal degrees of freedom associated with this $\mathbb Z_n$-symmetry.

The geometry $\conical$ is known to be dual to a state in the D1-D5 CFT.\footnote{see the references on p.8 in \cite{Balasubramanian:2014sra}.}
This state is identified by a twist field, which makes $N = pn$ fields $X^{1},\cdots,X^{N}$ subject to the condition that under a $2\pi$-rotation, they transform as follows ($k=0,2,\cdots p-1)$:
\begin{align}
	(X^{kn + 1},\cdots,X^{(k+1)n-1} ,X^{(k+1)n}) \rightarrow (X^{kn +2},\cdots, X^{(k+1)n},X^{kn+1}).
\end{align}
This reflects the global $\mathbb Z_n$-symmetry that corresponds to the $\mathbb Z_n$-gauge symmetry of the bulk.

The above theory can be embedded to a larger theory defined on $n$ times longer circle: $\theta\sim \theta + 2n\pi$.
The theory consists of $p$ single-valued fields $\tilde X^k$, each of which is obtained by glueing $X^{kn+1}, \cdots X^{k(n+1)}$ in order.
In this covering CFT, the original CFT is a sector which is symmetric under $2\pi$ times integer rotation.

The dual geometry of the covering CFT is still \eqref{eq: conical AdS}, except that $\theta$ runs over $n$ times larger region: $\theta\sim \theta + 2n\pi$.
The conical singularity has already been removed, and \eqref{eq: conical AdS} is completely equivalent to the pure AdS$_3$, meaning that the $\mathbb Z_n$-gauge is ungauged now.
In other words, the original bulk theory can be realized by gauging this covering geometry with the $\mathbb Z_n$-identification.
Thus, any quantity that descends to the original bulk theory must be $\mathbb Z_n$-symmetric in the larger bulk.

Note that in the covering CFT, the central charge is $\tilde c = c/n$, where $c$ is that of the original CFT, because the degrees of freedom is now $p = N/n$ instead of $N$.
According to this, the gravitational constants of the two bulk are related as $\tilde G = nG$.


\subsection*{The entanglement shadow}
Here we refere figure 2 in \cite{Balasubramanian:2014sra}.
There are $n$ geodesics that connects given two boundary points in $\conical$.
According to the Ryu-Takayanagi formula, the shortest one gives the entanglement entropy of the region between the two points.
It is easily confirmed by geodesic computations that the minimum geodesic cannot enter the region
\begin{align}
	r<L\cot(\pi/2n)/n =:r_\mathrm{c},
\end{align}
no matter how long is the interval between the two boundary points.
This is the entanglement shadow of \eqref{eq: conical AdS}, which cannot be described by the entanglement.
However, the other $n-1$ geodesics do enter the entanglement shadow.

In the covering geometry, on the other hand, each of these $n$ geodesics corresponds to a different boundary region; if the interval corresponding to the minimum geodesic is $[\theta_0,\theta_0+\Delta]$, then the other intervals corresponding to the rest geodesics are given as $[\theta_0,\theta_0+\Delta + 2m\pi]$.
(These regions are originally identical due to the $\mathbb Z_n$ gauge symmetry.)
Therefore, each of the long geodesics ascends to a minimum geodesic of its corresponding interval, which leads to the notion of the entwinement.


\subsection*{The entwinement}
The entwinement is defined on the covering CFT.
Let us consider an interval $R$ on the covering CFT and its $\mathbb Z_n$ translations $g_m R$, where $g_m$'s are $\mathbb Z_n$ generators.
The entwinement of the region $R$ is defined as
\begin{align}
	E_R = \sum_{m=1}^n S_{g_m R} = \sum_{m=1}^n g_m S_R = n S_R,
\end{align}
where $S_R$ is the conventional entanglement entropy in the large CFT.
Since $E_R$ is $\mathbb Z_n$-invariant, it can descends to $\conical$.

If the length of $R$ is $2\alpha\,(0\leq \alpha\leq n\pi)$, we have
\begin{align}
	S_R = \frac{\tilde c}{3}\ln \l(\frac{2L}{\mu} \sin \frac{\alpha}{n} \r),\label{eq: covering EE}
\end{align}
and hence with $c = 3L/(2G)$, we obtain
\begin{align}
	E_R = \frac{c}{3}\ln \l(\frac{2L}{\mu} \sin \frac{\alpha}{n} \r) 
	=  \frac{L}{2G}\ln \l(\frac{2L}{\mu} \sin \frac{\alpha}{n} \r)
	=:E(\alpha).
\end{align}

As mentioned above, there are $n$ geodesics for region $R$, and the lengths of them are given as
\begin{align}
	\ell(\phi + \pi m) = 2GL \ln \l(\frac{2L}{\mu}\sin \frac{\phi+\pi m}{n}\r),
\end{align}
where $2\phi\,(\phi\in [0,\pi])$ is the length of $R$.
(The minimum one is given by $m=0$.)
Thus with the identification of $\alpha = \phi + \pi m\,(\alpha\in [0,n\pi])$, we see
\begin{align}
	\ell(\alpha) = \frac{E(\alpha)}{4G}.
\end{align}
In the view point of $\conical$, the discussion here says that the entwinement is obtained by the naive continuation of the entanglement for $\alpha\in [0,\pi/2]$ to $\alpha \in [0,n\pi]$, and the entwinement version of the Ryu-Takayanagi formula is the equivalence between entwinements and geodesic lengths.

The entwinement reflects the internal degrees of freedom.
For simplicity, let us consider the case $p=1$.
In the covering CFT, \eqref{eq: covering EE} measures, for example, the entanglement between $X^1$ and $X^2$, the entanglement between internal degrees of freedom associated with $\mathbb Z_n$.
This is because if we take $[0,\pi]$ for $R$ in the covering space, then $R$ is for $X^1$ and $\bar R$ is for $X^{2,\cdots,n}$, by the definition of $\tilde X^1$.
Thus, $S_R$ measures how entangled $R$ is with $\bar R$, and $E_R$ includes such quantities in a gauge-invariant way.

\subsection*{The bulk reconstruction}
If we use the entwinement for the so-called hole-ography \cite{Balasubramanian:2013lsa,Czech:2014ppa},\footnote{See also \href{https://albertmcc.github.io/web/overview_papers/Hole_ographic_spacetime.pdf}{\color{blue}my overview 1} and \href{https://albertmcc.github.io/web/overview_papers/Nuts_and_Bolts_for_Creating_space.pdf}{\color{blue}my overview 2}} instead of the entanglement itself, entanglement shadows can be reconstructed by following the same way we usually do in the hole-ography.


%バルク再構築





































%%
\bibliographystyle{jhep} 
\bibliography{reference.bib}
\end{document}
