\documentclass[12pt]{article}

\usepackage{paper}

%commands
\renewcommand{\thefootnote}{\fnsymbol{footnote}}
\renewcommand{\theequation}{\arabic{equation}}
\newcommand{\overview}[1]{\href{#1}{\color{blue}my overview}}

\date{}

\begin{document}
{\Large\noindent
[Overview\footnote{
Written by Daichi Takeda (takedai.gauge@gmail.com)
}]
\hfill{\normalsize last updated: \today}
\\[2mm]
%title
\textbf{Integral Geometry and Holography\cite{Czech:2015qta}
}
}

\noindent
\hfill
\textbf{Bartlomiej Czech, Lampros Lamprou, Samuel McCandlish, James Sully}%author

\renewcommand{\thefootnote}{\arabic{footnote})}
\setcounter{footnote}{0}
\vspace{12pt}
%%body
Hole-ography, which originates in \cite{Balasubramanian:2013lsa}, is known as a method to reconstruct the bulk geometry in AdS/CFT correspondence.
In ref.\ \cite{Czech:2015qta}, they reinterpreted hole-ographic reconstruction formulas by a mathematic tool called \textit{kinematic space}, which is the space of oriented geodesics.
By that, the geometric information of a time slice in AdS$_3$ can be considered to be encoded in the kinematic space, which is intrinsically a property of the boundary theory.

As a warm-up, let us start with Euclidean $\mathbb R^2$.
All geodesics, lines, in this space are written in the form $x\cos \theta + y\sin\theta - p = 0$.\footnote{
Note that $|p|$ is the distance from the origin to the line.
}
The kinematic space is the space of all oriented geodesics, so we have to define the orientation of each line.
If we continuously change the line from $(\theta,p)$ to $(\theta+\pi,-p)$, we see that the line comes back to the original one, with the formal endpoints at infinity exchanged.
Therefore, we regard $(\theta+\pi,p)$ as the inversely oriented version of $(\theta,p)$.
Then the kinematic space is characterized by the pair $(\theta,p)\in \mathbb R^2$ with the identification $\theta\sim \theta + 2\pi$.

We can measure the length of an arbitrary curve $\gamma$ by calculating a volume form defined on the kinematic space $\mathcal K$.
Crofton formula gives it as
\begin{align}
	\mbox{length of $\gamma$}=\frac{1}{4}\int_{\mathcal K}\omega\,n_\gamma(\theta,p),\qquad
	\omega = \d\theta\wedge \d p,
	\label{eq: Crofton}
\end{align}
where $n_\gamma(\theta,p)$ is the number of times the line $(\theta,p)$ intersects with $\gamma$.
In the integral, $p$ runs over $\mathcal R$ and $\theta$ on $[0,2\pi]$.

They applied the above story to the context of AdS$_3$/CFT$_2$.
We consider a time slice $\Sigma$ on static asymptotically AdS$_3$ spacetime $M$ in global patch, and a curve $\gamma$ on $\Sigma$; the metric on $\partial M$ is $\d s^2 = -\d t^2 + L^2\d\theta^2$.
Let $[u,v]$ be an interval of $\theta$ on $\partial \Sigma = \Sigma\cap \partial M$, and $S(u,v)$ be the length of the bulk geodesic connecting $u$ and $v$.
Note that interval $[u,v]$ corresponds to the identical two geodesics having different orientations.
Their conjecture is that, if we replace $\omega$ in \eqref{eq: Crofton} with
\begin{align}
	\omega = \frac{\partial^2 S(u,v)}{\partial u\partial v}\d u\wedge \d v,
\end{align}
then eq.\eqref{eq: Crofton} holds on $\Sigma$.

The formula is correct  at least for any convex closed curve $\gamma$.
To show this, we define $\ell(u)(>0)$ such that the geodesic corresponding to $[u,u+\ell(u)]$ is tangent to $\gamma$.
Then, we see $n_\gamma(u,v) = 0$ for $v<u + \ell(u)$ and $n_\gamma(u,v) = 2$ for $v>u+\ell(u)$, and hence\footnote{
The other factor $2$ comes from the two types of the orientation.
}
\begin{align}
	\mbox{length of $\gamma$}
	&=\frac{2\cdot 2}{4}\int_0^{2\pi}\d u\,\int_{u+\ell(u)}^{u+\pi}\d v\,\frac{\partial^2 S}{\partial u\partial v}=-\int_0^{2\pi}\d u\, \l.\frac{\partial S(u,v)}{\partial u}\r|_{v = u+\ell(u)}.
	\label{eq: differential entropy}
\end{align}
Here we have used $(\partial_u S)|_{v = u\pm\pi} = 0$, which holds because the geodesic length is maximum when the interval length is $\pi$.\footnote{
Comment: this statement is not true when the bulk has a hole like a black hole, thus \textcolor{red}{the conjecture must be modified for holographic theories having such bulks.}}
Eq.\eqref{eq: differential entropy} is exactly the differential entropy formula (\overview{https://albertmcc.github.io/web/overview_papers/Hole_ographic_spacetime.pdf}) under Ryu-Takayanagi formula, and gives the right length of $\gamma$ according to \cite{Headrick:2014eia}.

From the strong subadditivity, we can show that $\partial_u\partial_v S$ is always positive.
The strong subadditivity is an inequality given by
\begin{align}
	S(AB) + S(BC) - S(B)-S(ABC) \geq 0.
\end{align}
If we choose $A,B$ and $C$ as
\begin{align}
	A=[u-\d u],\qquad
	B=[u,v],\qquad
	C = [v,v+\d v]
\end{align}
with $\d u,\d v>0$, then we have
\begin{align}
	S(u-\d u,v) + S(u,v+\d v) - S(u,v) - S(u-\d u,v+\d v)
	=
	\frac{\partial^2 S(u,v)}{\partial u\partial v}\d u\d v\geq 0.
\end{align}

Next, let us discuss how points on $\Sigma$ are interpreted in $\mathcal K$.
As considered in \cite{Czech:2014ppa} (\overview{https://albertmcc.github.io/web/overview_papers/Nuts_and_Bolts_for_Creating_space.pdf}), using shrinking limit of closed curves is useful.
If a closed curve $\gamma$ shrinks up to a point $A$, then for each $u$, only one $v$ makes geodesic $(u,v)$ intersect with $\gamma = A$.
Let $v_A(u)$ denote the critical $v$, and the curve $v= v_A(u)$ on $\mathcal K$ called ``point-curve."
Ref.\cite{Czech:2014ppa} have already given a conjecture to define point-curves from the boundary entanglement entropy, and in ref.\cite{Czech:2015qta}, the extension of it under the assumption that $\Sigma$ is Riemannian.

Finally, the distance between any two points on $\Sigma$ is also given in terms of $\mathcal K$.
Let $v_A(u)$ and $v_B(u)$ be point-curves on $\mathcal K$, and $\gamma_{AB}$ be the geodesic from $A$ to $B$.
As depicted in fig.10 in \cite{Czech:2015qta}, if $v$ satisfies
\begin{align}
	\mathrm{min}\{v_A(u),v_B(u)\}\leq v \leq \mathrm{max}\{v_A(u),v_B(u)\}
	\label{eq: vAvB}
\end{align}
for fixed $u$, then geodesic $(u,v)$ intersects with $\gamma_{AB}$ once, and does not intersect otherwise.\footnote{
If we choose $\gamma_{AB}$ as the other curves connecting $A$ and $B$, there is a curve which intersects with $\gamma_{AB}$ twice.}
Thus, from \eqref{eq: Crofton}, we conclude,
\begin{align}
	\mbox{geodesic length between $A$ and $B$} = 
	\frac{2}{4}\int_0^{2\pi}\d u\int_{\mathrm{eq}.\eqref{eq: vAvB}}\d v.
\end{align}































%%
\bibliographystyle{jhep} 
\bibliography{reference.bib}

\end{document}
