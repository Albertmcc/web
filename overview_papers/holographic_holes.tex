\documentclass[12pt]{article}

\usepackage{paper}

%commands
\renewcommand{\thefootnote}{\fnsymbol{footnote}}
\renewcommand{\theequation}{\arabic{equation}}
\newcommand{\overview}[1]{\href{#1}{\color{blue}my overview}}

\date{}

\begin{document}
{\Large\noindent
[Overview\footnote{
Written by Daichi Takeda (takedai.gauge@gmail.com)
}]
\hfill{\normalsize last updated: \today}
\\[2mm]
%title
\textbf{Holographic Holes and Differential Entropy\cite{Headrick:2014eia}
}
}

\noindent
\hfill
\textbf{Matthew Headrick, Robert C. Myers, Jason Wien}%author

\renewcommand{\thefootnote}{\arabic{footnote})}
\setcounter{footnote}{0}
\vspace{12pt}
%%body

Hole-ography is a method to reconstruct a curve on $t=\mathrm{const}$ and its length in pure AdS$_3$ from the differential entropy, which is a quantity defined by a family of the entanglement entropy of CFT$_2$ \cite{Balasubramanian:2013lsa} (see \href{https://albertmcc.github.io/web/overview_papers/Hole_ographic_spacetime.pdf}{\color{blue}my overview}).
Here $t$ is the time of the static coordinate.
In \cite{Headrick:2014eia}, they showed that the hole-ography can be also used for more generic bulk geometries subject to some assumptions, with curves not restricted to be on $t=\mathrm{const}$.

A spacelike geodesic was characterized by an $\theta$-interval, $[\theta-\alpha,\theta + \alpha]$, in \cite{Balasubramanian:2013lsa}.
However, if we want to describe curves not limited to be on $t = \mathrm{const}$,  $\theta$-intervals are not useful.
Thus we might better to express a geodesic by its boundary endpoints: $\gamma_L(\lambda),\gamma_R(\lambda)$.
Here $\lambda$ is a parameter labeling a family of the endpoint pairs to specify a spatial geodesic, and the family creates a spacelike curve in the bulk, as its envelop.
We call the bulk curve $c_\gamma = c_\gamma^M(\lambda)$, where $M$ expresses the bulk coordinates and the geodesic specified by $\gamma_L(\lambda)$ and $\gamma_R(\lambda)$ is tangent at $c_\gamma^M(\lambda)$.

The differential entropy is defined as\footnote{
This definition is equivalent to that of \cite{Balasubramanian:2013lsa} when geodesics are on $t = \mathrm{const}$, as explained in \cite{Headrick:2014eia}.
}
\begin{align}
	E = \int\d\lambda\,  \l.\frac{\partial S(\gamma_L(\lambda),\gamma_R(\lambda'))}{\partial \lambda'}\r|_{\lambda' = \lambda},
	\label{eq: differential entropy}
\end{align}
where $S(\gamma_L,\gamma_R)$ is the length of the geodesic from $\gamma_L$ to $\gamma_R$, which is dual to the boundary entanglement entropy, or more precisely the entwinement \cite{Balasubramanian:2014sra}.
The following statement was shown by them: \textit{given that the pair $(\gamma_L(\lambda),\gamma_R(\lambda))$ is periodic and smooth, then the length of $c_\gamma$ is equal to $E$.}\footnote{
Great attention should be paid to that the periodicity is assumed.
For example, BTZ black hole unfortunately breaks this assumption.
}

Let us follow the proof.
Length $S$ in \eqref{eq: differential entropy} is expressed as
\begin{align}
	S(\gamma_L(\lambda),\gamma_R(\lambda))
	=
	\int_{s_L}^{s_R}\d s\,\sqrt{(\dot x(s;\lambda),\dot x(s;\lambda))}
	\qquad
	(x(s_{L,R}) = \gamma_{L,R}(\lambda)),\label{eq: action}
\end{align}
where $(~,~)$ is the inner product defined by the bulk metric, $x(s,\lambda)$ is the geodesic from $\gamma_L(\lambda)$ to $\gamma_R(\lambda)$, and $\dot~ := \partial /\partial s$.
Since $S$ is of the ordinary action form,\footnote{
The reparameterization-invariance of $s$ and that it consists of up to the first derivative of the position.
}
we can use the knowledge of analytical mechanics.
The integrand in \eqref{eq: differential entropy} is rewritten as
\begin{align}
	 \l.\frac{\partial S(\gamma_L(\lambda),\gamma_R(\lambda'))}{\partial \lambda'}\r|_{\lambda' = \lambda}
	 &=
	 \gamma_R^\mu\,'(\lambda)
	  \frac{\partial S}{\partial \gamma_R^\mu}(\gamma_L(\lambda),\gamma_R(\lambda))\nonumber\\
	  &=
	  \gamma_R^M\,'(\lambda)
	  \frac{\partial S}{\partial \gamma_R^M}(\gamma_L(\lambda),\gamma_R(\lambda))\nonumber\\
	  &=
	  \frac{\partial x^M(s_R;\lambda)}{\partial\lambda} p_M(x(s_R;\lambda),\dot x(s_R;\lambda)),
\end{align}
where $p_M(x(s),\dot x(s))$ is the conjugate momentum of off-shell position $x^M(s)$. (If those are accompanied by ``$;\lambda$'', then they take the on-shell value defined above.) In the last equality, we have used the knowledge that the derivative of the on-shell action with the final position is the conjugate momentum at the position.

On the other hand, the length of $c_\gamma$, which we call $L$, is given as
\begin{align}
	L = 	\oint \d \lambda\,\sqrt{(c_\gamma'(\lambda),c_\gamma'(\lambda))}.
\end{align}
This is of the off-shell version of \eqref{eq: action}.
It is easy to show that equation
\begin{align}
	\sqrt{(\dot y(s),\dot y(s))} = \dot y^M(s) p_M(y(s),\dot y(s))\label{eq: Lagrangian}
\end{align}
holds in general (even in off-shell),\footnote{
This generally follows from the reparameterization-invariance, but can also be shown directly of course.
}
and hence,
\begin{align}
	L = \oint\d\lambda\, c_\gamma^M\,'(\lambda) p_M(c_\gamma(\lambda),c_\gamma'(\lambda)). 	
\end{align}
As $c_\gamma$ is the envelop of $\{\dot x(s;\lambda)\}$, at the tangent point $s = s_c$, we have
\begin{align}
	x(s_c;\lambda) = c_\gamma(\lambda),\qquad
	\exists\alpha(\lambda) > 0,~ \alpha(\lambda)\dot x(s_c;\lambda) = c_\gamma'(\lambda).
\end{align}
Note that $s_c$ can be chosen to be independent of $\lambda$ by reparameterizing $s$.
Then we obtain
\begin{align}
	L &= \oint\d\lambda\,\frac{\partial x^M(s_c;\lambda)}{\partial \lambda}p_M(x(s_c;\lambda),\alpha(\lambda)\dot x(s_c;\lambda))\nonumber\\
	&=
	\oint\d\lambda\,\frac{\partial x^M(s_c;\lambda)}{\partial \lambda}p_M(x(s_c;\lambda),\dot x(s_c;\lambda)),
\end{align}
where in the last equality, we have used follows from the property of the momentum that $p(y(s),\alpha \dot y(s)) = p(y(s),\dot y(s))$ following from \eqref{eq: Lagrangian}.

Therefore, the remaining task to accomplish is to show that
\begin{align}
	 E - L = \oint\d\lambda\,\l.\frac{\partial x^M(s;\lambda)}{\partial \lambda}p_M(x(s;\lambda),\dot x(s;\lambda))\r|^{s_R}_{x_c}
\end{align}
must vanish.
Using the relation between the on-shell action and the momenta of endpoints again, we can rewrite the above as
\begin{align}
	E - L = \oint\d\lambda\,\frac{\partial}{\partial \lambda}S(x(s_c;\lambda),x(s_R;\lambda)).
\end{align}
Since $x(s;\lambda)$ is also periodic about $\lambda$ by the assumption, we see this vanish.




















%%
\bibliographystyle{jhep} 
\bibliography{reference.bib}
\end{document}