\documentclass[12pt]{article}

\usepackage{paper}

%commands
\renewcommand{\thefootnote}{\fnsymbol{footnote}}
\renewcommand{\theequation}{\arabic{equation}}

\date{}

\begin{document}
{\Large\noindent
[Overview\footnote{
Written by Daichi Takeda (takedai.gauge@gmail.com)
}]
\hfill{\normalsize last updated: \today}
\\[2mm]
%title
\textbf{Entanglement Entropy and Quantum Field Theory\cite{Calabrese:2004eu}
}
}

\noindent
\hfill
\textbf{Pasquale Calabrese, John Cardy}%author

\renewcommand{\thefootnote}{\arabic{footnote})}
\setcounter{footnote}{0}
\vspace{12pt}
%%body
It is well-known that the entanglement entropy in CFT$_2$ can be calculated by applying \textit{replica trick}.
They introduced two local fields, which is later called \textit{branch-point twist fields} \cite{Cardy:2007mb}, so as to compute the R\' enyi entropy.
As the consequence, the trace of the $n$-th power of the reduced density matrix is expressed by the two-point correlation function of the two twisted fields, and hence the entanglement entropy is derived from the $n$-derivative of it.
Since now that our concern is the two-point correlator, the entanglement entropy of a compactified CFT$_2$ can also be calculated by applying the conformal transformation to the correlator.
I also referred review paper \cite{Calabrese:2009qy} to write the following.

Let $\phi$ be the fundamental field in the theory, then the partition function is written as follows:
\begin{align}
	Z = \int\mathcal D\phi\,\exp\l[-\int_{\mathbb C}\d^2 z\,\mathcal L[\phi]\r].
\end{align}
Here the complex coordinate $z$ is decomposed as $z = \sigma + i\tau$ with $\tau$ being the imaginary time.
We now consider the vacuum entanglement entropy of region
\begin{align}
	A = \{z\in\mathbb C\,|\,0 \leq \sigma\leq u,~\tau = 0\}
\end{align}
at the time slice $\tau = 0$.
In this case, we have
\begin{align}
	\mathrm{Tr}\rho_A^n &= \frac{Z_{\mathcal{R}_n}}{Z^n},\label{eq: Tr rho}\\
	Z_{\mathcal R_n} &= 
		\int_{\mathcal R_n}\mathcal D\phi\,\exp\l[-\int_{\mathcal R_n}\d^2 w\,\mathcal L[\phi]\r],
		\label{eq: R partition}
\end{align}
where $\mathcal R_n$ is the Riemann surface constructed by joining $n$ Riemann sheets at $A$ as the usual way, and $w$ is its coordinate.
Since the Lagrangian density is local, \eqref{eq: R partition} can be re-written as
\begin{align}
	Z_{\mathcal R_n} = \int_\mathrm{B.C.}\mathcal D\phi_1\cdots\mathcal D\phi_n\,
	\exp\l[-\int_{\mathbb C}\d^2 z\,(\mathcal L[\phi_1]+\cdots+\mathcal L[\phi_n])\r],
	\label{eq: n partition}
\end{align}
where B.C.\ denotes the boundary condition at $A$ on each sheet:
\begin{align}
	\mathrm{B.C.}:\phi_i(\sigma,+0) = \phi_{i+1}(\sigma,-0)\quad(\sigma\in A)\quad
	\mathrm{and}\quad
	\phi_i(\sigma,+0) = \phi_i(\sigma,-0)\quad(\sigma\not\in A). 
	\label{eq: BC}
\end{align}
Note that $i+n$-th and $i$-th sheet are identical on $\mathcal R_n$.

The partition function \eqref{eq: n partition} is now a path integral of $n$ fields $\phi_i\,(i=1,\cdots,n)$, each of which are defined on the same complex plane $\mathbb C$ with \eqref{eq: BC}.
We may expect that $Z_{\mathcal R_n}$ would be expressed by the operator insertions at $z=0$ and $z=u$ on $\mathbb C$, instead of using \eqref{eq: BC}.
The operators are branch-point twist operators, which we write as $\mathcal T_n(0),\bar{\mathcal T}_n(u)$ here.
Those operators are formally defined through
\begin{align}
	\braket{\mathcal T_n(0)\bar{\mathcal T}_n(u)}_{\mathbb C,\mathcal L^{(n)}}
	\propto
	\int_\mathrm{B.C.}\mathcal D\phi_1\cdots\mathcal D\phi_n\,
	\exp\l[-\int_{\mathbb C}\d^2 z\,\mathcal L^{(n)}[\phi_1,\cdots,\phi_n]\r],
	\label{eq: twist}
\end{align}
where
\begin{align}
	\mathcal L^{(n)}[\phi_1,\cdots,\phi_n](z)
	=
	\mathcal L[\phi_1](z)+\cdots+\mathcal L[\phi_n](z).
	\label{eq: n lagrangian}
\end{align}
The correlator of the l.h.s.\ in \eqref{eq: twist} is defined on the theory of $\phi_i$'s on $\mathbb C$ without B.C.
Twist operators $\mathcal T,\bar{\mathcal T}$ are in principle some composite operators of $\phi_i$'s.
Assuming that $\mathcal T,\bar{\mathcal T}$ are primary\footnote{
I could not understand whether or not twist operators do exist and are primary, but it seems reasonable to think it true according to \cite{Cardy:2007mb}.
}
having weight $d_n$, we obtain
\begin{align}
	\braket{\mathcal T_n(0)\bar{\mathcal T}_n(u)}_{\mathbb C,\mathcal L^{(n)}}\propto
	\frac{1}{u^{2d_n}}.
	\label{eq: TT}
\end{align}

To calculate the entropy, we need the vacuum expectation value $\braket{T(w)}_{\mathcal R_n,\mathcal L}$, where $T(w)$ is the energy momentum tensor of theory $(\mathcal R_n,\mathcal L)$.
Through map
\begin{align}
	z = \l(\frac{w}{w-u} \r)^{1/n},
\end{align}
the Riemann surface $\mathcal R_n$ is mapped to $\mathbb R$.
The energy momentum tensor $T(\zeta)$ of $(\mathbb R,\mathcal L)$ is related with $T(w)$ as
\begin{align}
	T(w) = \l(\frac{\partial \zeta}{\partial w}\r)^2T(z) + \frac{c}{12}\{\zeta,w\},
	\label{eq: T(w)}
\end{align}
where the Schwarzian derivative is
\begin{align}
	\{\zeta,w\} = \frac{\zeta''' \zeta'-(3/2)(\zeta'')^2}{(\zeta')^2},
\end{align}
and $c$ is the central charge.
Taking the vacuum expectation value of \eqref{eq: T(w)}, we obtain
\begin{align}
	\braket{T(w)}_{\mathcal R_n,\mathcal L} = \frac{c}{24}\l(1-\frac{1}{n^2}\r)\frac{u^2}{w^2(w-u)^2},
\end{align}
by using $\braket{T(\zeta)}_{\mathbb C,\mathcal L}=0$ which follows from the translational invariance.

On the other hand, \eqref{eq: n partition} and \eqref{eq: twist} means that $\braket{T(w)}_{\mathcal R_n,\mathcal L}$ can be computed as a correlator with the twist operators.
In concrete, we have 
\begin{align}
	\braket{T(z)}_{\mathcal R_n,\mathcal L} 
	= \frac{\braket{\mathcal T_n(0)\bar{\mathcal T}_n(u)T_j(z)}_{\mathbb C,\mathcal L^{(n)}}}
	{\braket{\mathcal T_n(0)\bar{\mathcal T}_n(u)}_{\mathbb C,\mathcal L^{(n)}}}
\end{align}
for $z$ describing $j$-th sheet.\footnote{
I used $w$ to describe the coordinate of $\mathcal R_n$, but now used $z$ to describe the coordinate of $\mathbb C$.
Note that $\mathcal R_n$ can be described by the pair $(j,z)$.
}
Here $T(w)$ corresponds to $T_j(w)$ which is defined by $\mathcal L[\phi_j]$ in theory $(\mathbb C,\mathcal L^{(n)})$.

Since the energy momentum tensor of the whole theory of $(\mathbb C,\mathcal L^{(n)})$, $T^{(n)}$, is given by the summation of $T_j$,\footnote{
Note that $T^{(n)}$ is defined by $\mathcal L^{(n)}$ of \eqref{eq: n lagrangian}.
}
we get
\begin{align}
	\frac{\braket{\mathcal T_n(0)\bar{\mathcal T}_n(u)T^{(n)}(z)}_{\mathbb C,\mathcal L^{(n)}}}
	{\braket{\mathcal T_n(0)\bar{\mathcal T}_n(u)}_{\mathbb C,\mathcal L^{(n)}}}
	=
	\frac{nc}{24}\l(1-\frac{1}{n^2}\r)\frac{u^2}{w^2(w-u)^2}.
\end{align}
From the usual formula\footnote{Since theory $(\mathbb C,\mathcal L^{(n)}$ must be rotational invariant, the}
\begin{align}
	&\braket{\mathcal T_n(a)\bar{\mathcal T}_n(b)T^{(n)}(z)}_{\mathbb C,\mathcal L^{(n)}}=\nonumber\\
	&\qquad
	\l(\frac{1}{z-a}\frac{\partial}{\partial a} + \frac{d_n}{(w-a)^2}
	+\frac{1}{z-b}\frac{\partial}{\partial b} + \frac{d_n}{(w-b)^2} \r)
	\braket{\mathcal T_n(a)\bar{\mathcal T}_n(b)}_{\mathbb C,\mathcal L^{(n)}}
\end{align}
with \eqref{eq: TT}, the weight now can be identified:
\begin{align}
	d_n = \frac{c}{12}\l(n-\frac{1}{n}\r).\label{eq: scaling}
\end{align}
Thus, combining \eqref{eq: Tr rho}, \eqref{eq: TT} and \eqref{eq: scaling}, we conclude
\begin{align}
	\mathrm{Tr}\rho_A^n = c_n\l(\frac{u}{a}\r)^{-c(n-1/n)/6},
\end{align}
where $c_n$ and $a$ are constants independent of $u$, in particular we have $c_1=1$ by $\mathrm{Tr}\rho_A = 1$.

The entanglement entropy of $A$ is 
\begin{align}
	S_A = \lim_{n\to 1}\frac{1}{1-n}\mathrm{Tr}\rho_A^n = \frac{c}{3}\ln\l(\frac{u}{a}\r) - c_1',
\end{align}
where $c_n'$ denotes $n$-derivative of $c_n$ ($c_1'$ is non-universal constant).

If the CFT of interest is on the cylinder obtained by compactifying $\sigma$-direction with length $L$, we use the map $\xi = (L/2\pi)\ln (-iz)$ to compute the entropy.
Applying the conformal transformation to $\braket{\mathcal T_n(a)\bar{\mathcal T}_n(b)}_{\mathbb C,\mathcal L^{(n)}}$ and putting $a/b = \exp(2\pi i v/L)$, we obtain
\begin{align}
	\mathrm{Tr}\rho_{A'}^n = c_n\l(\frac{L}{a\pi }\sin\frac{v\pi}{L}\r)^{-c(n-1/n)/6},\qquad
	\mathrm{i.e.}\qquad
	S_{A'} = \frac{c}{3}\log\l(\frac{L}{a \pi}\sin\frac{v\pi}{L}\r) - c_1'.
\end{align}
The entropy of thermalized CFT$_2$ can also be computed by following the same process.
In this case, we compactify the $\tau$-direction.






%%
\bibliographystyle{jhep} 
\bibliography{reference.bib}
\end{document}
